\begin{appendices}

\renewcommand{\theFancyVerbLine}{%
	\textcolor{red}{\small
		\arabic{FancyVerbLine}}}

\section{Squashed e-note model}
\label{appendix:squashed-e-note-model}

The squashed e-note model is a specialization of the Seraphis membership proof model that allows relatively simpler (and more efficient) proof structures.

First we will describe the model, then discuss how it satisfies relevant security requirements when applied to Seraphis.


\subsection{Model}
\label{appendix:squashed-e-note-model-model}

\begin{enumerate}
    \item Let $G_1,...,G_n$ be generators whose discrete logarithm relations with each other, with $G$, and with $H$ are unknown.

    \item Let $\mathbb{S}$ represent a set of tuples $\{K_i, C_i\}$, where\vspace{.115cm}
    \begin{align*}
        K_i &= z_i G + s_{i,1} G_1 + s_{i,2} G_2 + ... + s_{i,n} G_n \\
        C_i &= x_i G + a_{i,1} H
    \end{align*}

    \item Let $\mathbb{S}^t$ represent a set of `transformed' tuples $\{K^t_i, C^t_i\}$, where\vspace{.115cm}
    \begin{align*}
        K^t_i &= \mathcal{H}_6(K_i, C_i)*K_i \\
        C^t_i &= C_i
    \end{align*}

    \item Perform a range proof on each $C^t_i \in \mathbb{S}^t$ (recall Section \ref{subsubsec:confidential-transactions-range-proofs}).

    \item Let $\mathbb{Q}$ represent a set of squashed tuples $\{Q_i\}$, where\vspace{.115cm}
    \[Q_i = K^t_i + C^t_i\]

    \item Let $\tilde{S}$ represent a tuple $\{K', C'\}$, where\vspace{.115cm}
    \begin{align*}
        K' &= z' G + s'_1 G_1 + s'_2 G_2 + ... + s'_n G_n \\
        C' &= x' G + a'_1 H
    \end{align*}

    \item Let $\tilde{Q} = K' + C'$.

    \item Demonstrate that, within a security parameter $k$, $\tilde{Q}$ corresponds to some $Q_{\pi} \in \mathbb{Q}$, where $\pi$ is unknown to the verifier, such that:
    \begin{enumerate}
        \item The discrete log relation of $\tilde{Q} - Q_{\pi} = [(z' + x') - (\mathcal{H}_6(K_{\pi}, C_{\pi})*z_{\pi} + x_{\pi})]*G$ with respect to $G$ is known.
    \end{enumerate}

    \item Perform a range proof on $C'$.

    \item Demonstrate knowledge of $z', s'_1,...,s'_n$ such that $K' = z' G + s'_1 G_1 + s'_2 G_2 + ... + s'_n G_n$.
\end{enumerate}

The benefit of this specialization compared to the underlying membership proof model is you only need to prove the discrete log in one commitment to zero relation, rather than two. For example, with a SAG (e.g.\ LSAG \cite{Liu2004} without linking) or Groth/Bootle one-of-many proof on the set $\{\tilde{Q} - Q\}$. The efficiency implications are discussed in Section \ref{sec:efficiency}, which compares possible instantiations of Seraphis using the two models.


\subsection{Requirement satisfaction}
\label{appendix:squashed-e-note-model-req-satisfaction}

\subsubsection{Underlying membership proof model}

We argue that the squashed e-note model satisfies the underlying membership proof model.

Let $\mathbb{S}^t$ be the input to the underlying model. We will show that the following requirements, adapted from Section \ref{subsec:seraphis-membership proofs}, are met.

\begin{enumerate}
    \item Demonstrate that, within a security parameter $k$, $\tilde{S}$ corresponds to some $S^t_{\pi} \in \mathbb{S}^t$, where $\pi$ is unknown to the verifier, such that:
    \begin{enumerate}
        \item $s'_j == \mathcal{H}_6(K_{\pi}, C_{\pi})*s_{\pi,j}$ for $j \in 1,...,n$
        \item $a'_1 == a_{\pi,1}$
        \item The prover must have knowledge of $\mathcal{H}_6(K_{\pi}, C_{\pi})*z_{\pi}$.
    \end{enumerate}
\end{enumerate}

Observe the following.

\begin{enumerate}
    \item The prover must know $[(z' + x') - (\mathcal{H}_6(K_{\pi}, C_{\pi})*z_{\pi} + x_{\pi})]$ and $z'$ to satisfy the squashed e-note model, and $x'$ and $x_{\pi}$ to construct the range proofs on $C'$ and $C^t_{\pi}$. Therefore the prover must know $\mathcal{H}_6(K_{\pi}, C_{\pi})*z_{\pi}$. The point $\mathcal{H}_6(K_{\pi}, C_{\pi})$ is considered `public knowledge', so the prover must also know $z_{\pi}$.

    \item Range proofing $C_{\pi}$ and $C'$ means they have the form $x G + a H$, implying they contain no $G_1,...,G_n$ components. Therefore, demonstrating discrete log with respect to $G$ in the commitment to zero $\tilde{Q} - Q_{\pi}$ means it must be the case that $s'_j == \mathcal{H}_6(K_{\pi}, C_{\pi})*s_{\pi,j}$ for $j \in 1,...,n$.

    \item Suppose $K_i$ has the form $K_i = z_i G + s_{i,1} G_1 + s_{i,2} G_2 + ... + s_{i,n} G_n + b H$. Since the model demonstrates that $K'$ does not contain any $H$ components, and the commitment to zero $\tilde{Q} - Q_{\pi}$ means all non-$G$ components balance in those two points, it must be the case that $C' = x' G + (a_{i,1} + \mathcal{H}_6(K_{\pi}, C_{\pi})*b)*H$. However, the term $\mathcal{H}_6(K_{\pi}, C_{\pi})$ is both uniformly distributed and implicitly dependent on the values $a_{i,1}$ and $b$, so the value $a'_1 = (a_{i,1} + \mathcal{H}_6(K_{\pi}, C_{\pi})*b)$ will be uniformly distributed in $\mathbb{Z}_l$ (assuming $b$ is non-zero). Since the range proof on $C'$ means that $a'_1$ must be in the range $[0,...,2^z - 1]$, if $b \neq 0$ then the probability that a range proof on $a'_1$ can succeed is $2^z/l$. This means $a'_1 == a_{i,1}$ can be assumed to be true within the security parameter $k$ if $1/k > 2^z/l$.\footnote{Typically $l \approx 2^{252} - 2^{256}$, $2^z = 2^{64}$, and $k = 2^{128}$; $1/2^{128} > 2^{64}/2^{252}$ is true.} [[[formalize better?]]]
\end{enumerate}

\subsubsection{Seraphis structure}

\begin{enumerate}
    \item Note that a range proof on $C_i$ is equivalent to a range proof on $C^t_i$ since $C^t_i = C_i$.

    \item In Seraphis, range proofs on $C_i$ are created when e-notes are first constructed (i.e.\ as outputs of a transaction). This means transaction authors, who reference $C_i$ as part of transaction membership proofs, will not themselves construct range proofs on $C_i$. As a consequence, transaction authors won't necessarily know $x_{\pi}$.

    However, we do not consider this a security problem. When a transaction author sends an e-note to a recipient, they are `delegating spend authority' to that recipient. In the context of membership proofs (i.e.\ step 8 above), any person who knows $(\mathcal{H}_6(K_{\pi}, C_{\pi})*z_{\pi} + x_{\pi})$ must have learned that value by cooperating with the original transaction author. Therefore, whoever constructs a membership proof for an e-note in the above model must be acting as a `proxy' of that e-note's original author. Since the author knows $x_{\pi}$ (recall that they must have range proofed $C_{\pi}$), the value $\mathcal{H}_6(K_{\pi}, C_{\pi})*z_{\pi}$ can be derived from the combined knowledge of the author and his proxy.

    To gain further confidence in this roundabout security proof, consider the following.
    \begin{enumerate}
        \item An e-note's author cannot create a membership proof for that e-note (i.e.\ complete step 8 in the model) unless they know $\mathcal{H}_6(K_{\pi}, C_{\pi})*z_{\pi}$. This trivially follows from the fact they know $x_{\pi}$ as the one who range proved $C_{\pi}$, and they must know $\mathcal{H}_6(K_{\pi}, C_{\pi})*z_{\pi} + x_{\pi}$ as the one who performed step 8 in the above model.

        \item Suppose $p_1 + p_2 = \mathcal{H}_6(K_{\pi}, C_{\pi})*z_{\pi} + x_{\pi}$ ($p_1$ or $p_2$ could be zero). Let the e-note author know $x_{\pi}$ and $p_2$; let the prover of step 8 know $p_1$. In order to complete step 8 from above, the prover must learn $p_2$. Can the prover acquire the pair $p_1, p_2$ without collaborating with someone who knows $x_{\pi}$?

        [[[formal proof? this is giving me a lot of trouble]]]
    \end{enumerate}

    \item Seraphis linking tags are computed from the output of a membership proof, namely the point $K'$ in $\tilde{S}$. However, $K'$ in the squashed e-note model applied to Seraphis has the form $K' = t_k G + \mathcal{H}_6(K^o_{\pi}, C_{\pi})*[k^o_a*U + k^o_b*G]$. This means linking tags will have the form $\tilde{K} = (1/(\mathcal{H}_6(K^o_{\pi}, C_{\pi})*k^o_a))*G$ instead of $\tilde{K} = (1/k^o_a)*G$. Since $\mathcal{H}_6(K^o_{\pi}, C_{\pi})$ is uniquely defined by each e-note (and not malleable), these modified linking tags are also unique per e-note. In other words, only one linking tag can be produced for each e-note in the ledger.

    There is one interesting side-effect. The value $\mathcal{H}_6(K^o_{\pi}, C_{\pi})$ is dependent on the e-note commitment $C_{\pi}$, so it is possible for two e-notes with the same address $K^o$ to produce different linking tags if they have different commitments.

    \item Step 10 in the above model is automatically satisfied by Seraphis because $K'$ is passed as input to the ownership/unspentness proof system, which demonstrates knowledge of the per-generator discrete log relations of its inputs.
\end{enumerate}


\subsection{Practical considerations}
\label{appendix:squashed-e-note-model-practical-considerations}

\begin{enumerate}
    \item Transaction verifiers can pre-compute steps 3 and 5 from the above model for every e-note in the ledger. The squashed tuples $Q_i$ can be stored in anticipation of new transactions that may require them.

    \item In transaction chaining (Section \ref{subsec:implementers-other-recommendations}), only step 8 in the above model needs to be deferred (assuming $Q_i$ values have been precomputed, and range proofs on $C_i$ already exist).
\end{enumerate}



\section{Generalized Schnorr signatures}
\label{appendix:generalized-schnorr-signatures}

- generalized Schnorr signature
    - signing with multiple private keys in parallel (shared challenge)
    - concise aggregation on common base points (compressing private keys)
    - multi-base signing (same DL across multiple bases)



\section{Groth/Bootle one-of-many proofs}
\label{appendix:grootle-one-of-many-proofs}

- Grootle (parallel vs non-parallel)



\end{appendices}