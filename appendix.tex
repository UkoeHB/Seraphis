\begin{appendices}

\renewcommand{\theFancyVerbLine}{%
	\textcolor{red}{\small
		\arabic{FancyVerbLine}}}

\section{Squashed e-note model}
\label{appendix:squashed-e-note-model}

The squashed e-note model is a specialization of the Seraphis membership proof model that allows relatively simpler (and more efficient) proof structures.

First we will describe the model, then discuss how it satisfies relevant security requirements when applied to Seraphis.


\subsection{Model}
\label{appendix:squashed-e-note-model-model}

\begin{enumerate}
    \item Let $G_0 = H_0$, and assume the discrete logarithm of $H_1$ with respect to any of $G_1,...,G_n$ is unknown.

    \item Let $\mathbb{S}$ represent a set of tuples $[K_i, C_i]$, where\vspace{.115cm}
    \begin{align*}
        K_i &= z_i G_0 + s_{i,1} G_1 + s_{i,2} G_2 + ... + s_{i,n} G_n \\
        C_i &= x_i H_0 + a_{i,1} H_1
    \end{align*}

    \item Let $\mathbb{S}^t$ represent a set of `transformed' tuples $[K^t_i, C^t_i]$, where\vspace{.115cm}
    \begin{align*}
        K^t_i &= \mathcal{H}_6(K_i, C_i)*K_i \\
        C^t_i &= C_i
    \end{align*}

    \item Perform a range proof on each $C^t_i \in \mathbb{S}^t$ (recall Section \ref{subsubsec:confidential-transactions-range-proofs}).

    \item Let $\mathbb{Q}$ represent a set of squashed tuples $[Q_i]$, where\vspace{.115cm}
    \[Q_i = K^t_i + C^t_i\]

    \item Let $\tilde{S}$ represent a tuple $[K', C']$, where\vspace{.115cm}
    \begin{align*}
        K' &= z' G_0 + s'_1 G_1 + s'_2 G_2 + ... + s'_n G_n \\
        C' &= x' H_0 + a'_1 H_1
    \end{align*}

    \item Let $\tilde{Q} = K' + C'$.

    \item Demonstrate that, within a security parameter $k$, $\tilde{Q}$ corresponds to some $Q_{\pi} \in \mathbb{Q}$, where $\pi$ is unknown to the verifier, such that:
    \begin{enumerate}
        \item The discrete log relation of $\tilde{Q} - Q_{\pi} = [(z' + x') - (\mathcal{H}_6(K_{\pi}, C_{\pi})*z_{\pi} + x_{\pi})]*G_0$ with respect to $G_0$ is known.
    \end{enumerate}

    \item Perform a range proof on $C'$.

    \item Demonstrate knowledge of $z', s'_1,...,s'_n$ such that $K' = z' G_0 + s'_1 G_1 + s'_2 G_2 + ... + s'_n G_n$.
\end{enumerate}

The benefit of this specialization compared to the underlying membership proof model is you only need to prove the discrete log in one commitment to zero relation, rather than two. For example, with a SAG (e.g.\ LSAG \cite{Liu2004} without linking) or Groth/Bootle one-of-many proof on the set $\{\tilde{Q} - Q_i\}$. The efficiency implications are discussed in Section \ref{sec:efficiency}, which compares possible instantiations of Seraphis using the two models.


\subsection{Requirement satisfaction}
\label{appendix:squashed-e-note-model-req-satisfaction}

\subsubsection{Underlying membership proof model}

We argue that the squashed e-note model satisfies the underlying membership proof model.

Let $\mathbb{S}^t$ be the input to the underlying model. We will show that the following requirements, adapted from Section \ref{subsec:seraphis-membership proofs}, are met.

\begin{enumerate}
    \item Demonstrate that, within a security parameter $k$, $\tilde{S}$ corresponds to some $S^t_{\pi} \in \mathbb{S}^t$, where $\pi$ is unknown to the verifier, such that:
    \begin{enumerate}
        \item $s'_j == \mathcal{H}_6(K_{\pi}, C_{\pi})*s_{\pi,j}$ for $j \in 1,...,n$
        \item $a'_1 == a_{\pi,1}$
        \item The prover must have knowledge of $\mathcal{H}_6(K_{\pi}, C_{\pi})*z_{\pi}$.
    \end{enumerate}
\end{enumerate}

Observe the following.

\begin{enumerate}
    \item The prover must know $[(z' + x') - (\mathcal{H}_6(K_{\pi}, C_{\pi})*z_{\pi} + x_{\pi})]$ and $z'$ to satisfy the squashed e-note model, and $x'$ and $x_{\pi}$ to construct the range proofs on $C'$ and $C^t_{\pi}$. Therefore the prover must know $\mathcal{H}_6(K_{\pi}, C_{\pi})*z_{\pi}$. The point $\mathcal{H}_6(K_{\pi}, C_{\pi})$ is considered `public knowledge', so the prover must also know $z_{\pi}$.

    \item Range proofing $C_{\pi}$ and $C'$ means they have the form $x H_0 + a H_1$, implying they contain no $G_1,...,G_n$ components. Therefore, demonstrating discrete log with respect to $G_0$ in the commitment to zero $\tilde{Q} - Q_{\pi}$ means it must be the case that $s'_j == \mathcal{H}_6(K_{\pi}, C_{\pi})*s_{\pi,j}$ for $j \in 1,...,n$.

    \item Suppose $K_i$ has the form $K_i = z_i G_0 + s_{i,1} G_1 + s_{i,2} G_2 + ... + s_{i,n} G_n + e H_1$. Since the model demonstrates that $K'$ does not contain any $H_1$ components, and the commitment to zero $\tilde{Q} - Q_{\pi}$ means all non-$G_0$ components balance in those two points, it must be the case that $C' = x' H_0 + (a_{i,1} + \mathcal{H}_6(K_{\pi}, C_{\pi})*e)*H_1$. However, the term $\mathcal{H}_6(K_{\pi}, C_{\pi})$ is both uniformly distributed and implicitly dependent on the values $a_{i,1}$ and $e$, so the value $a'_1 = (a_{i,1} + \mathcal{H}_6(K_{\pi}, C_{\pi})*e)$ will be uniformly distributed in $\mathbb{Z}_l$ (assuming $e$ is non-zero). Since the range proof on $C'$ means that $a'_1$ must be in the range $[0,...,2^z - 1]$, if $e \neq 0$ then the probability that a range proof on $a'_1$ can succeed is $2^z/l$. This means $a'_1 == a_{i,1}$ can be assumed to be true within the security parameter $k$ if $1/k > 2^z/l$.\footnote{Typically $l \approx 2^{252} - 2^{256}$, $2^z = 2^{64}$, and $k = 2^{128}$; $1/2^{128} > 2^{64}/2^{252}$ is true.} [[[formalize better?]]]
\end{enumerate}

\subsubsection{Seraphis structure}

\begin{enumerate}
    \item Note that a range proof on $C_i$ is equivalent to a range proof on $C^t_i$ since $C^t_i = C_i$.

    \item In Seraphis, range proofs on $C_i$ are created when e-notes are first constructed (i.e.\ as outputs of a transaction). This means transaction authors, who reference $C_i$ as part of transaction membership proofs, will not themselves construct range proofs on $C_i$. As a consequence, transaction authors won't necessarily know $x_{\pi}$.

    However, we do not consider this a security problem. When a transaction author sends an e-note to a recipient, they are `delegating spend authority' to that recipient. In the context of membership proofs (i.e.\ step 8 above), any person who knows $(\mathcal{H}_6(K_{\pi}, C_{\pi})*z_{\pi} + x_{\pi})$ must have learned that value by cooperating with the original transaction author. Therefore, whoever constructs a membership proof for an e-note in the above model must be acting as a `proxy' of that e-note's original author. Since the author knows $x_{\pi}$ (recall that they must have range proofed $C_{\pi}$), the value $\mathcal{H}_6(K_{\pi}, C_{\pi})*z_{\pi}$ can be derived from the combined knowledge of the author and his proxy.

    To gain further confidence in this roundabout security proof, consider the following.
    \begin{enumerate}
        \item An e-note's author cannot create a membership proof for that e-note (i.e.\ complete step 8 in the model) unless they know $\mathcal{H}_6(K_{\pi}, C_{\pi})*z_{\pi}$. This trivially follows from the fact they know $x_{\pi}$ as the one who range proved $C_{\pi}$, and they must know $\mathcal{H}_6(K_{\pi}, C_{\pi})*z_{\pi} + x_{\pi}$ as the one who performed step 8 in the above model.

        \item Suppose $p_1 + p_2 = \mathcal{H}_6(K_{\pi}, C_{\pi})*z_{\pi} + x_{\pi}$ ($p_1$ or $p_2$ could be zero). Let the e-note author know $x_{\pi}$ and $p_2$; let the prover of step 8 know $p_1$. In order to complete step 8 from above, the prover must learn $p_2$. Can the prover acquire the pair $p_1, p_2$ without collaborating with someone who knows $x_{\pi}$?

        [[[formal proof? this is giving me a lot of trouble]]]
    \end{enumerate}

    \item Seraphis linking tags are computed from the output of a membership proof, namely the point $K'$ in $\tilde{S}$. However, $K'$ in the squashed e-note model applied to Seraphis has the form $K' = t_k G_0 + \mathcal{H}_6(K^o_{\pi}, C_{\pi})*[k^o_b*G_0 + k^o_a*G_1]$. This means linking tags will have the form $\tilde{K} = (1/(\mathcal{H}_6(K^o_{\pi}, C_{\pi})*k^o_a))*X$ instead of $\tilde{K} = (1/k^o_a)*X$. Since $\mathcal{H}_6(K^o_{\pi}, C_{\pi})$ is uniquely defined by each e-note (and not malleable), these modified linking tags are also unique per e-note. In other words, only one linking tag can be produced for each e-note in the ledger.

    There is one interesting side-effect. The value $\mathcal{H}_6(K^o_{\pi}, C_{\pi})$ is dependent on the e-note commitment $C_{\pi}$, so it is possible for two e-notes with the same address $K^o$ to produce different linking tags if they have different commitments.

    However, since $\mathcal{H}_6(K^o_{\pi}, C_{\pi})$ is `public knowledge', if two e-notes with the same one-time address are spent, then observers would see that they are both spent if they test the two linking tags with $\mathcal{H}_6(K^o_{x}, C_{x})\tilde{K}_x \stackrel{?}{=} \mathcal{H}_6(K^o_{y}, C_{y})\tilde{K}_y$. For this reason, we recommend mandating that all one-time addresses in the ledger be unique if the squashed e-note model is used.

    \item Step 10 in the above model is automatically satisfied by Seraphis because $K'$ is passed as input to the ownership/unspentness proof system, which demonstrates knowledge of the per-generator discrete log relations of its inputs.
\end{enumerate}


\subsection{Practical considerations}
\label{appendix:squashed-e-note-model-practical-considerations}

\begin{enumerate}
    \item Transaction verifiers can pre-compute steps 3 and 5 from the above model for every e-note in the ledger. The squashed tuples $Q_i$ can be stored in anticipation of new transactions that may require them.

    \item In transaction chaining (Section \ref{subsec:implementers-other-recommendations}), only step 8 in the above model needs to be deferred (assuming $Q_i$ values have been precomputed, and range proofs on $C_i$ already exist).
\end{enumerate}



\section{Composition proofs with Schnorr}
\label{appendix:composition-with-schnorr}

In this Appendix is our recommended approach to satisfying the Seraphis ownership/unspentness proof requirements (Section \ref{subsec:seraphis-ownership-unspentness-proofs}). First we lay out the proof system that satisfies those requirements, next we will describe a proof structure in that system, and finally we will apply the Fiat-Shamir transform \cite{fiat-shamir-transform} to that structure.


\subsection{Composition proof system}
\label{appendix:composition-proof-system}

[[[better terminology than `proof system'?]]]

\begin{enumerate}
    \item Assume there is a group point $K = x G_0 + y G_1$.

    \item Let\vspace{.115cm}
    \begin{align*}
        K_{t1} &= (1/y)*K \\
        \tilde{K} &= (1/y)*X \\
        K_{t2} &= K_{t1} - G_1
    \end{align*}

    \item Demonstrate the simultaneous discrete log of $K_{t1}$ and $\tilde{K}$ with respect to $K$ and $X$. A simultaneous discrete log proof shows knowledge of a single scalar that is the discrete log between two pairs of group elements (in this case $(1/y)$). [[[formalize better?]]]

    \item Demonstrate the discrete log of $K_{t2} = (x/y)*G_0$ with respect to $G_0$.
\end{enumerate}

\subsubsection{Seraphis requirements satisfaction}

[[[explain how it satisfies the requirements?]]]

\subsection{Composition proof structure}
\label{appendix:composition-proof-structure}

[[[better terminology than `proof structure'?]

Our proof structure is a Schnorr-like $\Sigma$-protocol between prover and verifier. Notably, for the discrete log of $K_{t2}$ from the proof system, we use the concise approach from \cite{clsag-eprint} to reduce proof sizes when constructing multiple proofs in parallel (i.e.\ reduce the number of responses required).

\begin{enumerate}
    \item Suppose the prover has keys $[x_i, y_i, K_i]$ for $i \in 1,...,n$, where $K_i = x_i G_0 + y_i G_1$.

    \item The prover generates random scalars $\alpha_0, \alpha_1, ..., \alpha_n \in_R \mathbb{Z}_l$.

    \item The prover computes $\alpha_0 G_0$, and $\alpha_i K$, $\alpha_i X$, $K_{t1,i} = (x_i/y_i)*G_0$, and $\tilde{K}_i = (1/y_i)*X$ for $i \in 1,...n$. He sends all of those to the verifier along with the keys $K_i$.

    \item The verifier generates a random challenge $c \in_R \mathbb{Z}_l$ and sends it to the verifier.

    \item The prover computes responses $r_0, r_1, ..., r_n$ and sends them to the verifier.\vspace{.115cm}
    \begin{align*}
        r_0 &\equiv \alpha_0 - c*(\sum^n_{i=1} \mathcal{H}_7(i, K_1,...,K_n)*(x_i/y_i)) \\
        r_i &\equiv \alpha_i - c*(1/y_i)
    \end{align*}

    \item The verifier checks the following equalities. If any of them fail, then the prover has failed to satisfy the composition proof system.\vspace{.115cm}
    \begin{align*}
        \alpha_0 G_0 &== r_0 G_0 + c*(\sum^n_{i=1} \mathcal{H}_7(i, K_1,...,K_n)*(K_{t1,i} - G_1)) \\
        \alpha_i K_i &== r_i K_i + c*K_{t1,i} \\
        \alpha_i X   &== r_i X + c*\tilde{K}_i
    \end{align*}
\end{enumerate}


\subsection{Non-interactive composition proofs}
\label{appendix:noninteractive-composition-proofs}

Here we apply the Fiat-Shamir transform \cite{fiat-shamir-transform} to the proof structure just described.

\subsubsection{Non-interactive proof}

\begin{enumerate}
    \item Suppose the prover has keys $x_i, y_i$, and $K_i$ for $i \in 1,...,n$, where $K_i = x_i G_0 + y_i G_1$.

    \item The prover generates random scalars $\alpha_0, \alpha_1, ..., \alpha_n \in_R \mathbb{Z}_l$.

    \item The prover computes $\alpha_0 G_0$, and $\alpha_i K$, $\alpha_i X$, $\tilde{K}_i = (1/y_i)*X$, and $K_{t1,i} = (x_i/y_i)*G_0$ for $i \in 1,...n$. Let the values $G_0, G_1, X, K_i, K_{t1,i}$, and $\tilde{K}_i$ be recorded in a global reference string $R$.

    \item The prover computes the challenge:
    \[c = \mathcal{H}_8(R, [\alpha_0 G_0], [\alpha_1 K_1], [\alpha_1 X],...,[\alpha_n K_n], [\alpha_n X])\]

    \item The prover computes responses $r_0, r_1, ..., r_n$.\vspace{.115cm}
    \begin{align*}
        r_0 &\equiv \alpha_0 - c*(\sum^n_{i=1} \mathcal{H}_7(i, K_1,...,K_n)*(x_i/y_i)) \\
        r_i &\equiv \alpha_i - c*(1/y_i)
    \end{align*}
\end{enumerate}

The composition proof is the tuple $\sigma_{cp} = [c, r_0, r_1, ..., r_n, G_0, G_1, X, K_1, K_{t1,1}, \tilde{K}_1, ..., K_n, K_{t1,n}, \tilde{K}_n]$.

\subsubsection{Verification}

The verifier does the following given a proof $\sigma_{cp}$.

\begin{enumerate}
    \item Compute the challenge:
    \begin{align*}
        c' = \mathcal{H}_8(&R, [r_0 G_0 + c*(\sum^n_{i=1} \mathcal{H}_7(i, K_1,...,K_n)*(K_{t1,i} - G_1))],\\
        &[r_1 K_1 + c*K_{t1,1}], [r_1 X + c*\tilde{K}_1],...,[r_n K_n + c*K_{t1,n}], [r_n X + c*\tilde{K}_n])
    \end{align*}
    \[\]

    \item If $c == c'$ then the proof is valid.
\end{enumerate}


\end{appendices}